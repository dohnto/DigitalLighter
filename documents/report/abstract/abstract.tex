The aim of this report is to describe the 13 weeks long process of developing the mobile application called Digital Lighter as well as the management techniques used to control the four membered team. This project represents the main assignment for students of the course TDT4290 -- Customer Driven Project held at Norwegian University of Science and Technology. 

The main purpose of the resulting application is to entertain the audience of the music concerts by enabling them to actively participate the common experience by using their mobile devices. The users are instructed to hold their mobile devices and raise their hands while becoming "pixels" in the huge "human-screen" which can be used to display various imagery.

During planning the research focusing on similar products was done, the development methodology was chosen and the collaboration tools were established. The requirements that gave rise to the basic application concept were then elicited. The implementation itself is based on Google Android platform while Java SDK was used. Besides standard Android API the open source computer vision library OpenCV was also utilized for necessary image processing. Other important technologies this project builds on comprise of netlib networking library and NTP protocol used for synchronizing the devices.

The application was incrementally improved during development process and the resulting product achieved all the requirements the customer specified. The conclusion can also be drawn that even though this project represents proof of a concept task the final version of the application is working product which given certain limitations can be scaled up in order to be used for more real world purposes.