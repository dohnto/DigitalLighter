The aim of this report is to describe the 13 weeks long process of developing the mobile application called Digital Ligheter as well as the management techniques used to control the four membered team. This project represents the main assignment for students of the course TDT4290 - Customer Driven Project held at Norwegian University of Science and Technology. 

The main purpose of the resulting application is to entertain the audience of the music concerts by enabling them to actively participate the common experience by using their mobile devices. The users are instructed to hold their mobile devices and raise their hands while becoming "pixels" in the huge "human-screen" which can be used to display various imagery.

The report gives an insight into the details of planning, preliminary research and requirements elicitation that gave rise to the basic concept which is used to gradually build on. The application is based on Android platform while Java SDK was used. Besides standard Android API the open source computer vision library OpenCV was also utilized for necessary image processing. Other important technologies this project builds on comprise of netlib networking library and NTP protocol used for synchronizing the devices.

The application 

The conclusion can also be drawn that


This is a proof of concept task covering all main problems of the domain.



--- State the problem, your approach and solution, and the main contributions of the paper. Include little if any background and motivation. Be factual but comprehensive. 
--- Uses an introduction-body-conclusion structure
--- purpose, methods, scope, results, conclusions, and recommendations.


- motivation + background (what and why are we doing)
- used methods

- describe your result
- conclusion (what we achieved, maybe mention chapter with reflection?)