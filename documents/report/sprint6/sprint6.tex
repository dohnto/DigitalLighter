\section{Sprint planning}
It was formerly intended to devote the last sprint

Due to the fact the final version of the project must be submitted on the fixed date this sprint is shorter by two work days and the workload must be adjusted accordingly. The team has eight work days to its disposal which makes it for 140 person-hours for the whole team.

All the documentation related stories for sprint 6 are presented in Table \ref{tab:sprint6Documentationstories}. 
%\caption{User stories selected for Sprint 1.}
\label{tab:sprint1Documentationstories}
\def\arraystretch{1.25}
 
\begin{longtable}{ccXcc}

\toprule[0.5mm]
\multirow{2}{*}{\textbf{ID}} &
\multirow{2}{*}{\textbf{Ref.}} & \multirow{2}{*}{\textbf{Description}} & \multicolumn{2}{c}{\textbf{Hours}} \\
 					& & & \textbf{Est.} & \textbf{Sp.} \\
%\textbf{ID} 	& \textbf{Description} 									& \textbf{Est.} & \textbf{Sp.} \\
\midrule
% === DOCUMENTATION ==========================
\textbf{343} 	&& {\bf As a student I have to work on Project Plan.} 	& 		12	& \textbf{12} \\
							
				
\hline
				&& \textbf{SUM:}		&		?	& \textbf{?}
 \\																			
\bottomrule[0.5mm]
\end{longtable}
 All the project management related stories for sprint 6 are presented in Table \ref{tab:sprint6storiesProcess}.
\begin{table*}[!ht]%\caption{User stories selected for Sprint 6.}
\def\arraystretch{1.25}
 
 \caption{Documentation stories selected for sprint 6}
 \label{tab:sprint6storiesProcess}

\begin{tabularx}{\textwidth}{ccXcc} 

\toprule[0.5mm]
\multirow{2}{*}{\textbf{ID}} &
\multirow{2}{*}{\textbf{Ref.}} & \multirow{2}{*}{\textbf{Description}} & \multicolumn{2}{c}{\textbf{Hours}} \\
 					& & & \textbf{Est.} & \textbf{Sp.} \\

%\textbf{ID} 	& \textbf{Description} & \textbf{Est.} & \textbf{Sp.} \\
\midrule


	
\textbf{P6.1} 	&
	\refwbs{wbs_project_management}{WBS 7.1.1}& {\bf As a student I have attend the meeting with the customer} 			& 	?	& \textbf{?} \\
	
\textbf{P6.2} 	&
	\refwbs{wbs_project_management}{WBS 7.1.2}& {\bf As a student I have to attend the meeting with the supervisor} 		& 	?	& \textbf{?} \\

\textbf{P6.3} 	&& {\bf  As a student I have to prepare for the final presenation} 		& 	?	& \textbf{?} \\

\textbf{P6.4} 	&& {\bf  As a student I have to deploy the application on app store} 	& 	?	& \textbf{?} \\
\textbf{P6.5} 	&& {\bf  As a student I have to prepare for submission of the report} 	& 	?	& \textbf{?} \\
\textbf{P6.6} 	&& {\bf  As a student I have to create final demo-video} 	& 	?	& \textbf{?} \\
							
\hline
				&& \textbf{$\sum$}		&		?	& \textbf{?}
 \\																			
\bottomrule[0.5mm]
\end{tabularx}
\end{table*}


\subsection{Duration}
%This sprint is 2 weeks long. From 28th of October 2013 to 10th of November 2013. We agreed
%on the date of presentation and showing the running demo – on Thursday 7th of November 2013.
%Estimated velocity is 240 hours since we agreed on 30 working hours per person per week. 

\section{Sprint Goal}

\section{System Burndown}

\section{Architecture}
\section{Implementation}
\section{Testing}
\section{Occurring risks}
\section{Customer feedback}
- agreed that we fulfilled all the requirements

\section{Retrospective}
This section reflects on the past sprint. In order to learn from the mistakes done and thus to improve the workflow it is necessary to answer two essential questions: "What went well" and "What could be improved".

\subsection{What went well}
\subsection{What could be improved}
