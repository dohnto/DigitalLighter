\section{Sprint planning}
It was formerly intended to devote the last sprint to the tasks related with finishing and refining the report as well as preparing the final presentation and rehearsing. Nevertheless as the application performance issues occurring during the last sprint prevented the team from accomplishing to record the final demonstration video part of the total sum of person-hours available in this sprint must be spent on creating the video again.

The planning of this sprint started already during the last customer meeting held on 7th November where he suggested to use different types of imagery to be displayed on borrowed phones (see Section \ref{txt:sprint5_customerfeedback}). One of the tasks therefore should be to create suitable images and/or videos in multiple scales as it was uncertain how many phones would be available during recording. The team members should also try to collect as many mobile devices as possible and also borrow the video recording device.

Next significant part of the sprint should be devoted to creating the demonstration video which would fulfill the requirement to amaze and entertain the viewers. As already planned this sprint will also mainly focus on finalizing and refining the report, preparing it for the submission, creating the final presentation and rehearse.

It should not be forgotten to deploy both server and client side application on Google Play as this is one of the non-functional requirements (see Section \ref{txt:requirements}).

Due to the fact the final version of the project must be submitted on the fixed date this sprint is shorter by two work days and the workload must be adjusted accordingly. The team has eight work days to its disposal which makes it for 140 person-hours for the whole team.

All implementation related stories for sprint 6 are presented in Table \ref{tab:sprint6stories}.
\begin{table*}%\caption{User stories selected for Sprint 3.}
 \def\arraystretch{1.25}
 \caption{Implementation user stories selected for sprint 3}
   \label{tab:sprint3stories}
 
\begin{tabularx}{\textwidth}{ccXcc}

\toprule[0.5mm]
\multirow{2}{*}{\textbf{ID}} &
\multirow{2}{*}{\textbf{Ref.}} & \multirow{2}{*}{\textbf{Description}} & \multicolumn{2}{c}{\textbf{Hours}} \\
 					& & & \textbf{Est.} & \textbf{Sp.} \\
%\textbf{ID} 	& \textbf{Description} 		& \textbf{Est.} & \textbf{Sp.} \\
\midrule
\textbf{I3.1} 	& \refreq{M4}	& {\bf As a server I need to be able to process images from the real world.}		& 32		& \textbf{28} \\

\textbf{I3.2} 	& \refwbs{wbs_testing}{WBS 6.2}	& {\bf As a server I need to be able to process images from the virtual world.}		& 20		& \textbf{23} \\

\textbf{I3.3} 	&\refreq{M4} 	& {\bf As a server I need to be able to detect one client from its light.} 		& 40		& \textbf{46} \\


\textbf{I3.4} 	& \refreq{M4}	& {\bf As a server I need to detect multiple clients from light.}		 &  5	& \textbf{8} \\


\textbf{I3.5} 	& \refreq{M4}	& {\bf As a server I need to map all available devices to grid.} 			 & 10 & \textbf{10} \\	

\midrule
		
				&& \textbf{$\sum$}		&		107	& \textbf{115}
 \\																			
\bottomrule[0.5mm]
\end{tabularx}
\end{table*}
All the documentation related stories for sprint 6 are presented in Table \ref{tab:sprint6Documentationstories}. 
%\caption{User stories selected for Sprint 1.}
\label{tab:sprint1Documentationstories}
\def\arraystretch{1.25}
 
\begin{longtable}{ccXcc}

\toprule[0.5mm]
\multirow{2}{*}{\textbf{ID}} &
\multirow{2}{*}{\textbf{Ref.}} & \multirow{2}{*}{\textbf{Description}} & \multicolumn{2}{c}{\textbf{Hours}} \\
 					& & & \textbf{Est.} & \textbf{Sp.} \\
%\textbf{ID} 	& \textbf{Description} 									& \textbf{Est.} & \textbf{Sp.} \\
\midrule
% === DOCUMENTATION ==========================
\textbf{343} 	&& {\bf As a student I have to work on Project Plan.} 	& 		12	& \textbf{12} \\
							
				
\hline
				&& \textbf{SUM:}		&		?	& \textbf{?}
 \\																			
\bottomrule[0.5mm]
\end{longtable}
 All the project management related stories for sprint 6 are presented in Table \ref{tab:sprint6storiesProcess}.
\begin{table*}[!ht]%\caption{User stories selected for Sprint 6.}
\def\arraystretch{1.25}
 
 \caption{Documentation stories selected for sprint 6}
 \label{tab:sprint6storiesProcess}

\begin{tabularx}{\textwidth}{ccXcc} 

\toprule[0.5mm]
\multirow{2}{*}{\textbf{ID}} &
\multirow{2}{*}{\textbf{Ref.}} & \multirow{2}{*}{\textbf{Description}} & \multicolumn{2}{c}{\textbf{Hours}} \\
 					& & & \textbf{Est.} & \textbf{Sp.} \\

%\textbf{ID} 	& \textbf{Description} & \textbf{Est.} & \textbf{Sp.} \\
\midrule


	
\textbf{P6.1} 	&
	\refwbs{wbs_project_management}{WBS 7.1.1}& {\bf As a student I have attend the meeting with the customer} 			& 	?	& \textbf{?} \\
	
\textbf{P6.2} 	&
	\refwbs{wbs_project_management}{WBS 7.1.2}& {\bf As a student I have to attend the meeting with the supervisor} 		& 	?	& \textbf{?} \\

\textbf{P6.3} 	&& {\bf  As a student I have to prepare for the final presenation} 		& 	?	& \textbf{?} \\

\textbf{P6.4} 	&& {\bf  As a student I have to deploy the application on app store} 	& 	?	& \textbf{?} \\
\textbf{P6.5} 	&& {\bf  As a student I have to prepare for submission of the report} 	& 	?	& \textbf{?} \\
\textbf{P6.6} 	&& {\bf  As a student I have to create final demo-video} 	& 	?	& \textbf{?} \\
							
\hline
				&& \textbf{$\sum$}		&		?	& \textbf{?}
 \\																			
\bottomrule[0.5mm]
\end{tabularx}
\end{table*}


\subsection{Duration}
This sprint is 10 days (8 work days) long. From 11th of November 2013 to 20th of November 2013. The final presentation of the project will be held on Thursday 21st of November 2013.
Estimated velocity is 160 hours since we agreed on 25 working hours per person per week. 

\section{Sprint Goal}

\section{System Burndown}

\section{Architecture}
\section{Implementation}
\section{Testing}
\section{Occurring risks}
\section{Customer feedback}
- agreed that we fulfilled all the requirements

\section{Retrospective}
This section reflects on the past sprint. In order to learn from the mistakes done and thus to improve the workflow it is necessary to answer two essential questions: "What went well" and "What could be improved".

\subsection{What went well}
\subsection{What could be improved}
