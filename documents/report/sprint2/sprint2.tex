In this chapter the planning and work-flow regarding Sprint 2 will be described. 
Everything from setting our goals to implementation, and testing. At the end we will evaluate whole sprint, and try to answer on following questions: What went well? What could be improved? What should we start doing? 

\section{Sprint planning}
The customer was very satisfied with the video for sprint 1, and suggested recording our future prototypes as well. 
After the successful sprint 1 review we started planning next sprint, sprint 2. Since the customer was very happy with our sprint 1 delivery, he gave us an opportunity to decide for ourselves what we wanted to work on this next sprint, and then ask for his approval. 
We decided to focus on multiple client support, and since sprint 1 just focused on connecting one client to the server, then we would also get the opportunity to refine the code we had implemented so far. The customer thought this was a really good idea, because working on this now could help us in future. Catching problems in early phases is much better then discovering them too close to the deadline. Then we might not get the needed time to fix it. Therefore Sprint 2 will focus on adding support for multiple clients connection and sending different signals to different clients. We should revise and improve our code so that we can move on to another core part, image processing, during the next sprint.


In the supervisor meeting the day after the sprint review, we showed our supervisor the project report. He was satisfied with the new structure, but he wanted us to finish more sections. He also wanted us so make a work break down structure for the project. There were also some chapters that could be merged together, which means we should refine the structure of the report. In this sprint we plan to work on these things. We also have to write the chapter for sprint 2.

\subsection{Duration}
This Sprint will be 2 weeks long. From 16.09.2013 to 29.09.2013.
We agreed on the date of presentation and showing the running demo - Thursday 26.09.2013.
Estimated velocity is 240h since we agreed on 30 working hours per person per week.

\subsection{User-stories}
\LTXtable{\textwidth}{sprint2/stories.tex}

\section{Sprint Goals}
Our goal for Sprint 2 is to deliver a working demo with a more refined core client-server module. 
This still includes registering services, listening for the client and sending simple signals to the client from the server application, but with a more refined code. Of course the client should still scan for the services, connect, receive signals, and play the commands. 
The goal is so sill use the established simple communication protocol. All of the above is extended so it works for several clients connected to the server.

Goals about the report is to refine the report and write the sections needed to be up to speed with the report.This means that we have to finish writing sprint 1, and write the sprint 2 chapter. 

\section{System Burndown}
\section{Architecture}
\section{Implementation}
\section{Testing}


\section{Occurring risks}

The risk table 3.3, taken from the risk management section in the planning chapter, shows many different risk. 
For this sprint most of them did not occur. 
This is because the hardest technical parts, which is the image processing, is not scheduled until the upcoming sprints. 
The user stories for this sprint was very clear, and the customer did not change any of them in the last minute. The team also did a good job with the time estimation. All of this avoids a lot of issues, which is good. 

The first risk in the risk table describes team members being sick. This did not occur, but what did occur was that that some of team members had to be absent in some of the working hours. 
The team solved this by having these team members work more independently, and with more flexible hours.  

\section{Retrospective and Evaluation}
In this section the team will take a look back at sprint 2, and review the sprint. The focus will be on what went well, and goal we achieved. Another focus will also be what we could have done differently, and of course and learn from what we could have done differently.

\subsection{Pros}
The team reached the sprint 2 goals. The team delivered the demo with our more refined code, and as planned this code supported multiple clients. The clients were able to play the commands the server sent. We also reached the milestone, Obedient crowd - Prototype 2.

The work done in the report is extremely good. The report is better structured now, and the team managed to finish the sections we planned for. We also spent a couple of extra hours to go through the whole report. This was to see if we needed to add more in the specific sections, or if we should remove something.  

Even though some member were had other responsibilities, and had to absent in some of the working hours, the team still got everything done in time for the sprint review. Everyone worked really well independently, as well as in team. 

The sprint planning in this sprint went really well. We had a perfect amount of work, and the hours estimated for each task, was very realistic, so that was good. 


\subsection{Cons}
The time tracking of each task could have been done better. 
Even though we track our time, the targetProcess3 program dose not have a feature for manually edit the time tracking. 
This means if we forget to do it right away, then the burn down chart wont be as reliable. Since this shows our time tracking we should try to be better at that.

When there is work that needs to be done in the report, the person who writes this section should read terminology, or what other team members have written earlier. This will help to get a better flow for the reader.
This will also save us a couple of hours when we have to refine the report later. 