In this chapter planning and work-flow regarding Sprint 1 will be described. 
All from setting our goals to implementation and testing. On the end we will evaluate whole sprint and try to answer on following questions: What went well? What could be improved? What should we start doing?  

\section{Sprint Planning}
After assembling all the tools in Sprint 0, we decided to start with the implementation of core modules.
As our understanding of task improved, we were able to come up with user stories from the perspective of user, customer, developer and student.
All user-stories were given to the customer so they can be prioritized. 
All but user-stories concerning our school obligations, like writing project plan, minutes, meetings with supervisor and attending lectures.
We decided to have school tasks as a user stories so we can better keep our time tracking. 
School user-stories were mandatory and already added as a user-stories of sprint 1.
On Monday 02.09.2013. we had the meeting with the customer where we estimated time we need for every user story.
Result of that meeting was list of rest of the user-stories for sprint 1.

\subsection{Duration}
This Sprint will be 2 weeks long. From 02.09.2013 to 15.09.2013.
We agreed on the date for presentation and showing the running demo - Thursday 12.09.2013.
Estimated velocity is 240h since we agreed on 30 working hours per person per week.

\subsection{Sprint 1 User-stories}

All Sprint 1 user-stories are presented in table \ref{tab:sprint1stories}

\LTXtable{\textwidth}{sprint1/stories.tex}

\section{Sprint Goal}

Our goal for this Sprint is to deliver working demo over core client-server module.
This includes registering services, listening for the client and sending simple signals to the client from the server application.
Scanning for the services, connecting, receiving signals and play commands on the client.
Establishing simple communication protocol and implementing Test-Flight in both applications, the server and the client, is goal of this sprint too.

\section{Architecture}

For the core communication and organization of our product we selected Client-Server Architecture \ref{fig:sprint1_arhitecture}.
Choosing this architecture was very intuitive to do as our project should consist of two applications and overall tasks should be partitioned. 
Client application should be able to light up different sequences of lights depending on server command signals.
And server application should be responsible for awaiting incoming clients connections, mapping clients to the grid and providing command play signals to the clients by broadcast.
Communication should be established over wireless network using same router for both applications, the user and the server, and without need for third party server nor Internet connection. 

\subsection{"4+1" view model}
For illustrating software architecture "4+1" Architectural Model View will be used.

\subsubsection{Logical View}
A description of the functional requirements of the architecture. The system is decomposed into a set of key abstractions, taken(mostly) from the problem domain, in the form of objects or object classes.

\begin{figure}[!ht]
	\centering
		\includegraphics[width=7cm]{sprint1/sprint1_class.png}
	\caption{Sprint 1 Architecture}
	\label{fig:sprint1_class}
\end{figure}

\subsubsection{Physical View}

\subsubsection{Process View}

\subsubsection{Development View}


\begin{figure}[!h]
	\centering
		\includegraphics[width=7cm]{sprint1/arhitecture.png}
	\caption{Sprint 1 Architecture}
	\label{fig:sprint1_arhitecture}
\end{figure}


\begin{figure}[!h]
	\centering
		\includegraphics[width=7cm]{sprint1/communication.png}
	\caption{Sprint 1 Communication}
	\label{fig:sprint1_communication}
\end{figure}





\section{Implementation}

Our first research of technologies that could help as at achieving client-server communication, without writing everything from scratch, was Bonjour software. 
It is the Apple's implementation of Zero configuration network (Zeroconf). 
Zero-configuration network (Zeroconf)  is a methodology and a set of special technologies that automatically create a usable computer network based on the Internet Protocol Suite (TCP/IP) when computers or network peripherals are interconnected. 
It does not require manual operator intervention or special configuration servers.
It is assembled of technologies that includes service discovery, addressing assignment, and host name resolution.
Aldo good and very useful tool it is not supported on Android, and since Android is our platform of choice we had to research further. 
We learned that we want zeroconf library with similar set of services as Bonjour written in Java.


Further research brought us to JmDNS. Java implementation of multi-cast DNS that can be used for service registration and discovery in Local Area Network. 
It works on most JDK1.6 compatible Virtual Machines, it comes as a library and it is easy to integrate with Android. 
JmDNS fulfill all of our expectations, but while learning about it over internet, we found that Android itself have built in Nestwork Service Discovery(NSD) that do the same thing.


Android Network Service Discovery(NSD) is supported from API version 16. 
It allows users to identify other devices on the local network, register services, broadcast connection information, scan for registered services and connect.
Even with minimum API limitation it is a part of Android platform, no third part libraries are needed, it will evolve with Android and therefore always be working.
Min API version was not of concern to the customer as he replied on mail regarding this question, so we made a decision of using Android Network Discovery(NSD) for client-server discovery and communication.

NSD send look like this:

Device Identifier: HelloWorldServer
Service Name: \_helloWorld.\_tcp.
Port: 27812
Network Address: 192.168.1.31

explain sockets

NSD
\section{Testing}

\section{Occurring risks}

That Android will not get NSD able for low APIs. 
Then we have to use third part libraries.

\section{Retrospective}
\subsection{Pros}
\subsection{Cons}
\section{Evaluation}
